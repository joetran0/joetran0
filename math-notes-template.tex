\documentclass[11pt]{article}

% Packages
\usepackage{amsmath, amssymb, amsthm, mathrsfs, enumitem, fancyhdr, tikz, xcolor, hyperref}
\usepackage[margin=1in]{geometry}
\hypersetup{
    colorlinks=true,
    linkcolor=blue,
    citecolor=blue,
    urlcolor=blue
}

% Header/Footer
\pagestyle{fancy}
\fancyhf{}
\lhead{Joe Tran}
\rhead{Math Notes}
\cfoot{\thepage}

% Theorem Environment
\newtheorem{theorem}{Theorem}[section]
\newtheorem{lemma}[theorem]{Lemma}
\newtheorem{proposition}[theorem]{Proposition}
\newtheorem{corollary}[theorem]{Corollary}
\theoremstyle{definition}
\newtheorem*{definition}{Definition}
\newtheorem*{example}{Example}
\newtheorem*{remark}{Remark}

% Shortcuts
\renewcommand{\AA}{\mathcal{A}}
\newcommand{\BB}{\mathcal{B}}
\newcommand{\CC}{\mathcal{C}}
\newcommand{\DD}{\mathcal{D}}
\newcommand{\EE}{\mathcal{E}}
\newcommand{\FF}{\mathcal{F}}
\newcommand{\GG}{\mathcal{G}}
\newcommand{\HH}{\mathcal{H}}
\newcommand{\II}{\mathcal{I}}
\newcommand{\JJ}{\mathcal{J}}
\newcommand{\KK}{\mathcal{K}}
\newcommand{\LL}{\mathcal{L}}
\newcommand{\MM}{\mathcal{M}}
\newcommand{\NN}{\mathcal{N}}
\newcommand{\OO}{\mathcal{O}}
\newcommand{\PP}{\mathcal{P}}
\newcommand{\QQ}{\mathcal{Q}}
\newcommand{\RR}{\mathcal{R}}
\renewcommand{\SS}{\mathcal{S}}
\newcommand{\TT}{\mathcal{T}}
\newcommand{\UU}{\mathcal{U}}
\newcommand{\VV}{\mathcal{V}}
\newcommand{\WW}{\mathcal{W}}
\newcommand{\XX}{\mathcal{X}}
\newcommand{\YY}{\mathcal{Y}}
\newcommand{\ZZ}{\mathcal{Z}}

\newcommand{\A}{\mathbb{A}}
\newcommand{\B}{\mathbb{B}}
\newcommand{\C}{\mathbb{C}}
\newcommand{\D}{\mathbb{D}}
\newcommand{\E}{\mathbb{E}}
\newcommand{\F}{\mathbb{F}}
\newcommand{\G}{\mathbb{G}}
\renewcommand{\H}{\mathbb{H}}
\newcommand{\I}{\mathbb{I}}
\newcommand{\J}{\mathbb{J}}
\newcommand{\K}{\mathbb{K}}
\renewcommand{\L}{\mathbb{L}}
\newcommand{\M}{\mathbb{M}}
\newcommand{\N}{\mathbb{N}}
\renewcommand{\O}{\mathbb{O}}
\renewcommand{\P}{\mathbb{P}}
\newcommand{\Q}{\mathbb{Q}}
\newcommand{\R}{\mathbb{R}}
\renewcommand{\S}{\mathbb{S}}
\newcommand{\T}{\mathbb{T}}
\newcommand{\U}{\mathbb{U}}
\newcommand{\V}{\mathbb{V}}
\newcommand{\W}{\mathbb{W}}
\newcommand{\X}{\mathbb{X}}
\newcommand{\Y}{\mathbb{Y}}
\newcommand{\Z}{\mathbb{Z}}
\renewcommand{\subset}{\subseteq}
\newcommand{\e}{\varepsilon}
\renewcommand{\phi}{\varphi}

\title{\textbf{Course Notes} \\[0.5em] \large York University}
\author{Joe Tran}
\date{\today}

\begin{document}

\maketitle

\tableofcontents

\newpage

\section{Preliminaries}

\begin{definition}
    A \textit{normed linear space} is a vector space $X$ over $\K$ equipped with a norm $\|\cdot\|$.
\end{definition}

\begin{example}
    The space $\ell_p(\N)$, for $1 \leq p < \infty$, is defined by
    \begin{equation*}
        \ell_p(\N) = \left\{x = (x_n)_{n = 1}^{\infty} \subseteq \K : \sum_{n = 1}^{\infty} |x_n|^p < \infty\right\}
    \end{equation*}
    with the norm $\|x\|_p = \left(\sum_{n = 1}^{\infty} |x_n|^p\right)^{\frac{1}{p}}$.
\end{example}

\section{Linear Operators}

\begin{definition}
    A linear operator $T : X \to Y$ between normed linear spaces is \textit{bounded} if there exists $M > 0$ such that
    \begin{equation*}
        \|Tx\|_Y \leq M\|x\|_X
    \end{equation*}
    for all $x \in X$.
\end{definition}

\begin{theorem}[Uniform Boundedness Principle]
    Let $\{T_{\alpha}\}_{\alpha \in A}$ be a family of bounded linear operators from $X$ to $Y$. If for all $x \in X$, the set $\{\|T_{\alpha}x\|\}_{\alpha \in A}$ is bounded, then
    \begin{equation*}
        \sup_{\alpha \in A} \|T_{\alpha}\| < \infty.
    \end{equation*}
\end{theorem}
    
\end{document}